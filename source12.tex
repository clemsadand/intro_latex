\documentclass[a4paper,12pt]{report}
\usepackage[utf8]{inputenc}
\usepackage[T1]{fontenc}
\usepackage[french]{babel}
\usepackage{amsmath, amssymb}
\newcommand{\abs}[1]{\lvert#1\rvert}%la valeur absolue
\title{Algèbre linéaire}%Définir titre du document
\author{Clément A.\\ Etudiant en MFA1}%Définir le nom de l'auteur
\date{\today}%Définir une date, la commande \today donne la date aujourd'hui
\begin{document}
\maketitle%Permet d'afficher le titre, le nom et la date au début du document

\chapter{Espaces vectoriels}
\section{Généralités}

\subsection{Définition}

Soit E un ensemble muni non vide muni d'une loi de composition interne +
et d'une loi de composition externe $\cdot$. On dit que E est un espace
vectoriel lorsque\dots

\subsection{Sous-espaces vectoriels}
Un sous-ensemble d'un espace vectoriel est appelé sous-espace vectoriel \dots
\end{document}