\documentclass[a4paper,12pt]{report}
\usepackage[utf8]{inputenc}
\usepackage[T1]{fontenc}
\usepackage[french]{babel}
\usepackage{amsmath, amssymb}
\newcommand{\abs}[1]{\lvert#1\rvert}%la valeur absolue
\usepackage{amsthm}
\newtheorem{definition}{Définition}[section]
\newtheorem{propriete}{Propriété}
\newtheorem{exemple}{Exemple}[definition]
\begin{document}
\section{Environnements numérotés}

\begin{definition}[Inverse]
On appelle inverse d'un nombre réel non nul $a$ le nombre $\frac{1}{a}$.
\end{definition}

\begin{propriete}[de Pythagore]
Dans un triangle rectangle, le carré de l'hypoténuse est la somme des
carrés des longueurs des deux autres côtés.
\end{propriete}

\begin{exemple}
ABC est un triangle rectangle en A tel que : $AB=30$ et $BC=50$.

Déterminer AC.
\end{exemple}
\end{document}
